\section{Leadership}

The governing body of the Guild shall be known as the CSGG Council (``the
Council''). It is composed of executive officers and representatives to
Department committees as described below.

\subsection{Executive Officers}\label{sec:officers-duties}

\textit{Executive Officers} (``officers'') are members of the Council empowered to represent and pass resolutions to the benefit of all Guild members. Officers shall be familiar with the Constitution, act impartially of their personal interests, and ensure a smooth transition to their successors. Additionally, Officers shall serve as described in~\ref{sec:communal-spaces}.

A duly-elected officer is one who was elected or acclaimed in their own right,
as opposed to being appointed in an interim capacity. Unless otherwise stated,
offices may be held by only one member, but members may hold multiple offices.
Seniority follows in the order listed below.

\subsubsection{President}

A \textit{President} shall preside over the Council, ensure smooth operation of
the Guild, and act as the spokesperson and represent of the Guild to the
Department. Duties include,

\begin{enumerate}
      \item Delegating and overseeing the Council and representatives on Department
            committees (\ref{sec:representatives});
      \item Chairing (or delegating) Council meetings (\ref{sec:council-meetings}).
      \item Appointing or staffing members to fill vacant roles where an election would not
            be appropriate;
      \item Collaborating with the Treasurer (\ref{sec:treasurer}) in managing the
            financial affairs of the Guild (\ref{sec:finances});
      \item Collaborating with Administrators (\ref{sec:administrators}) in communicating
            constitutional/procedural matters;
      \item Organizing general meetings and electronic votes; and
      \item Delivering a President's Report to the membership in April, May, or June.
\end{enumerate}
The President shall hold signing authority of financial accounts and expenses.
The President may not simultaneously hold either of the offices of
Vice-President or Treasurer.

\subsubsection{Vice-President}

A \textit{Vice-President (VP)} shall assist the President. In the absence (but
not vacancy) of the President, the VP shall act in their stead. In case of
vacancy of the Presidency, the VP succeeds as in~\ref{sec:vacancy}. The VP may
not simultaneously hold of the office of President.

\subsubsection{Treasurer}\label{sec:treasurer}

The \textit{Treasurer} shall modulate the financial business of the Guild,
maintaining long-term financial stability and ensuring reasonable spending.
Duties include,

\begin{enumerate}
      \item Reviewing and approving budget and reimbursement requests from fellow officers
            and Guild members;
      \item Drafting an annual budget (\ref{sec:finances}) based on input from the Council
            and general membership;
      \item Presenting a financial statement (even if brief) to each Council meeting; and
      \item Coordinating with the Department and Ambassador (\ref{sec:ambassador})
            regarding funding and grants.
\end{enumerate}
The Treasurer shall hold signing authority of financial accounts and expenses. The Treasurer may not simultaneously hold the office of President.

\subsubsection{Ambassador to the Union}\label{sec:ambassador}

An \textit{Ambassador to the Union} shall represent the interests of Guild
members before the GSU.\@ Duties include,

\begin{enumerate}
      \item Attending GSU Council meetings and, if relevant, present Guild members' issues;
      \item When appropriate, soliciting feedback from the membership on the GSU Council's
            activities;
      \item Advising Guild members concerning GSU activities (e.g.\ the student health/drug
            plan); and
      \item Carrying out the transfer of operating funds (i.e.\ the head grant) from the
            GSU to the Guild.
\end{enumerate}

\subsubsection{Workers’ Union Delegate}
A \textit{Workers' Union Delegate} shall represent the interests of Guild
members before the Canadian Union of Public Employees Local 3902 (CUPE 3902)
and regarding its Unit 1 jobs, including but not limited to teaching assistant,
course instructor, and invigilator positions within and beyond the Department.
Duties include,
\begin{enumerate}
      \item Attending CUPE 3902 Steward Council meetings and, if relevant, present Guild
            members' issues;
      \item Reporting to Guild members relevant information and activities of CUPE 3902 and
            Unit 1 jobs;
      \item Advising Guild members in matters concerning Unit 1 jobs, bargaining, the
            current Collective Bargaining Agreement (CBA) and grievance processes; and
      \item If necessary, mediating with Unit 1 hiring coordinators on behalf of Guild
            members.
      \item If necessary, conferring with Stewards for other departments.
      \item If the Delegate is not also the Steward for the Department, they (or their
            designate) must meet privately with the Steward at least once per academic
            term.
\end{enumerate}
The Delegate must themselves be a member of CUPE 3902 Unit 1 but not necessarily a Steward.

\subsubsection{Systems Administrators}\label{sec:administrators}

The Guild's \href{https://www.cs.toronto.edu/csgsbs/}{website},
\href{https://support.cs.toronto.edu/}{Computer Science Laboratory (CSLab)
      accounts}, mailing lists and other internet resources (such as
\href{https://drive.google.com/}{Google Drive}) shall be maintained by
\textit{Systems Administrators}. Information and communications should be
timely and accurate. The number of Administrators is not to exceed two ($2$).

\subsubsection{Social Coordinators}\label{sec:coordinators}

\textit{Social Coordinators} shall organize social activities, open and advertised to all Guild members. Coordinators may have specialized roles, such as Pubmaster or snack procurement. The Department may collaborate with CSGG to organize some events. The number of Coordinators is not to exceed six ($6$).

\subsubsection{Liaison for Applied Computing Students}

A \textit{Liaison for Applied Computing Students} shall represent the (unique)
interests and concerns of Guild members enrolled in the MScAC degree
(\ref{sec:members}). The Liaison must themselves be a member currently enrolled
in or graduated from the MScAC degree (e.g.\ pursuing a PhD after MScAC).

\subsection{Representatives to Department Committees}\label{sec:representatives}

Where the Department deems appropriate, the Guild maintains student
\textit{representatives} to Department committees. These roles are privy to
Council meetings but do not receive a vote in procedure. Department committee
representatives are nominated by the President or VP and approved by a majority
of the Council in a Council meeting or electronic vote. Appointments must be
officially communicated to the Department in a timely manner.

\subsubsection{Graduate Affairs Committee Representatives}

The Guild is entitled to two permanent seats on the Department's Graduate
Affairs Committee (GAC). At least one seat must be held by a PhD or MSc student
in the funded cohort. These representatives must report to the Council on a
monthly basis.

\subsection{Resignation \& Dismissal}

\subsubsection{Resignation}

An officer or representative may resign their post at any time by notifying the
senior-most officer in writing.

\subsubsection{Impeachment of Officers}\label{sec:impeachment}

Officers may be impeached and removed from the Council for reasons including,

\begin{enumerate}
      \item Repeatedly failing to fulfill duties or meet requirements of their post without
            leave or explanation;
      \item Failing to respond to communications within a reasonable time-frame without
            leave or explanation;
      \item Demonstrably acting in bad faith or abusing their office for personal reasons;
      \item Serious mistreatment of Guild members; or
      \item Otherwise bringing the Guild or Department into disrepute.
\end{enumerate}

Any Guild member may recommend an officer (``the Subject'') for impeachment by
notifying another officer in writing. Should one officer second this
recommendation, the Council shall hold an inquiry with the Subject. The Council
(including the Subject) shall then vote in a Council meeting or electronic
vote.

\begin{enumerate}
      \item If at least two-thirds ($66\%$) of the present Council votes to impeach, the
            Subject is dismissed from their role and~\ref{sec:vacancy} applies.
      \item Otherwise, the acquitted Subject is deemed acquitted and immune from
            impeachment for 60 days.
\end{enumerate}

\subsubsection{Recall of Representatives}\label{sec:recall}

Department committee representatives may be recalled by a majority vote
($50\%$) of the Council to refresh or diversify the Guild's representation.
Representatives can also be dismissed by a majority vote ($50\%$) of the
Council for any reasons described under~\ref{sec:impeachment}.

\subsubsection{Vacancies \& Succession}\label{sec:vacancy}
Handling vacancies in the offices of President or VP is straightforward:
\begin{enumerate}
      \item If the office of President is vacant, the duly-elected VP becomes the
            President.
      \item If the office of VP is vacant, the duly-elected President may appoint a VP with
            consent of Council.
      \item If the offices of both the President and VP are vacant, the senior-most
            duly-elected officer acts as President in the interim and call a by-election
            for all vacant roles within 15 days.
\end{enumerate}
If any other office is vacant, the President must within 15 days call a by-election according to~\ref{sec:elections}. In the meantime, the President by choose to appoint an interim officer (who is also eligible for by-election).

\begin{enumerate}
      \item If the by-election's nomination period passes and the interim officer for some
            office is the only nominee, they become duly-elected to that office by
            acclamation.
      \item If some office is without eligible nominees when the nomination period ends or
            is otherwise left vacant after a by-election, the President may appoint a new
            officer (considered duly-elected) with consent the Council. The President may
            choose to leave the office vacant until the next general election.
\end{enumerate}