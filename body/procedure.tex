\section{Operations \& Procedures}

\subsection{Communications}\label{sec:communications}
Announcements and communications will be made through,
\begin{itemize}
      \item The graduate students' mailing list;
      \item The Department's Slack workspace
            (\href{https://csuoft-grads.slack.com}{\texttt{csuoft-grads.slack.com}}); and
      \item The Guild's website
            (\href{https://www.cs.toronto.edu/csgsbs}{\texttt{www.cs.toronto.edu/csgsbs}})
            (event pages and major announcements);
      \item Optionally, the Guild's Discord server
            (\href{https://discord.gg/qKWCNFvNBF}{\texttt{discord.gg/qKWCNFvNBF}}).
\end{itemize}
It is the responsibility of the Council to keep the student body informed through all appropriate media.

\subsubsection{Contacting the Officers}
The official mailing address for the Guild is
\href{mailto:csgsbs@cs.toronto.edu}{\texttt{csgsbs@cs.toronto.edu}}. Individual
Officers may also be contacted through email, Slack or Discord, whichever is
appropriate.

\subsubsection{Additional Documents}
The following information shall be on the website for members to view at any
time.

\begin{itemize}
      \item The Constitution in its active form.
      \item A list of current Officers and their positions (voting and non-voting) along
            with contact information.
\end{itemize}

\subsection{Communal Spaces}\label{sec:communal-spaces}
The Guild shall operate an office in room \textbf{BA 2283} in the
\href{https://map.utoronto.ca/?id=1809#!m/494470}{Bahen Centre for Information
      Technology}.

\begin{enumerate}
      \item Unless excused by the Council, every Officer must serve at least one hour per
            week as the Guild's representative in BA 2283. A schedule must be made
            available on the website~(\ref{sec:communications}).
      \item All members shall hold access to BA 2283 upon request to the Department.
\end{enumerate}
Secondary communal spaces include the faculty/staff lounge in \textbf{BA 5248}
and any room reserved for events.

\subsection{General Membership Meetings}
\textit{General Membership Meetings (GMMs)} serve as a medium for comprehensive communications and an open forum for questions and feedback from the general membership. Officer elections and constitutional amendments may also take place in GMMs.

\begin{enumerate}
      \item There must be at least one general meeting in each of the Fall and Winter
            terms.
      \item The President or VP may call a GMM with at least 7-days notice.
      \item No GMM may be held within 15 days of another.
      \item Social Coordinators (\ref{sec:coordinators}) shall assist with organizing GMMs
            as necessary.
\end{enumerate}

\subsubsection{Setting}
GMMs may be held in-person (recommended) or online. In the case of a in-person
GMM,

\begin{enumerate}
      \item The Council must, upon request, make the GMM hybrid-accessible through
            videoconferencing;
      \item The GMM must take place between 9:00AM and 9:00PM in an accessible campus
            setting; and
      \item The President, VP, or a Social Coordinator must reserve a space for the GMM.\@
\end{enumerate}

\subsubsection{Quorum}\label{sec:quorum}
Quorum for general meetings shall be 15 or one-twentieth ($5\%$) (whichever is
higher) of current active membership of the Guild (graduate students who are
not on leave). No official business may take place until quorum is established.

\subsubsection{Agenda}
The President or VP (whoever called the GMM) sets the agenda, which must begin
with establishing quorum and informing members of the current version of the
Constitution.

\subsection{General Membership Votes}\label{sec:voting}
Voting on the budget, elections or any motions may take place during GMMs or
via online voting.

\begin{enumerate}
      \item All relevant materials must be made available to members
            (\ref{sec:communications}) at least 7 days before the end of voting.
      \item Online votes must be open for at least 7 days and represent at least one-tenth
            ($10\%$) of the general membership.
      \item Votes during a GMM with quorum established (\ref{sec:quorum}) may take place by
            show-of-hand or electronic vote.
      \item Should a GMM not reach quorum, votes on the agenda shall be moved to online
            voting within 15 days.
      \item Unless otherwise specified, votes are by simple majority ($50\%$).
\end{enumerate}

\subsection{Council Meetings}\label{sec:council-meetings}
The Council shall meet at least once per month to discuss upcoming events and
ongoing priorities. The President or their designate (usually the VP) shall set
the agenda and \textit{chair} the meeting.

\subsubsection{Attendance of General Members}
Any Guild member may request to attend the next Council meeting. This request
must be reviewed and approved by the chair;\@ if it is to be rejected, the
reviewer must provide a reason to the requestor and the Council.

\subsubsection{Motions}
Any member in attendance may present motions before the Council. Motions are
voted on by Officers and are adopted with, unless otherwise specified
(e.g.~\ref{sec:impeachment},~\ref{sec:recall}), a simple majority ($50\%$)
(``consent of Council'').

\subsection{Finances}\label{sec:finances}
The fiscal year shall begin in October of the current academic year to
September of the next academic year. At the beginning of every academic year
(September/October), the Treasurer shall prepare and present a budget to the
general membership and be passed as described in~\ref{sec:voting}.

\subsubsection{Spending Limits}
By default, the annual spending limit that can be budgeted will be $C - L - S$,
where,
\begin{enumerate}
      \item $C$ is the amount of cash available in all banking accounts;
      \item $L$ is the sum of any outstanding liabilities, including uncashed cheques; and
      \item $S$ is the amount spent in in the previous year.
\end{enumerate}
If the limit is calculated as being less than the projected expenses for the year, the maximum spending that can be budgeted can then be the sum of:

\begin{itemize}
      \item One-tenth ($10\%$) of the initial funds from all accounts;
      \item Precisely known sources of incoming for the new year; and
      \item One and a twentieth ($105\%$) of the previous year's income from imprecisely
            known sources for the new year.
\end{itemize}

\subsection{Elections}\label{sec:elections}
Every March, a general election shall be held at a date determined by the
President on the advice of the Council. If required by~\ref{sec:vacancy},
further by-elections may be held throughout the year to fill vacant positions.
Information, including dates, must be communicated (\ref{sec:communications}).

\subsubsection{Nominations}
A nomination period of at least 7 days must be held before voting begins.
Nominees must either be current members or become members within 45 days of the
end of the voting period (e.g.\ an incoming graduate student). Previously
impeached members (\ref{sec:impeachment}) require a majority approval ($50\%$)
of the Council to be nominated.

\subsubsection{Campaigning}
After nominations, the Council must provide a medium for candidates to address
the Guild membership either during a GMM or via email. Although typically
unnecessary, the President may optionally hold a campaigning period before
voting of no more than 7 days. Candidates may campaign in any ``reasonable''
manner in accordance with University and Department policies. Gross violations
may result in disqualification by a two-thirds ($66\%$) vote of Council.

\subsubsection{Elections Voting}
Elections may take place in a GMM or online voting as described
in~\ref{sec:voting}.

\begin{enumerate}
      \item Ballots and other listings of candidates must be in alphabetical order by
            surname.
      \item For contested positions (involving more than one candidate), voters may select
            one candidate or spoil their vote (i.e.\ select ``none of the above'').
      \item Uncontested positions (involving only one candidate) will come down to a
            ``yes''/``no'' vote.
      \item Once polls have closed, vote shall be tallied by two current Officers and any
            volunteer scrutineers. Any candidate must recuse themselves from this process.
      \item Results of the election must then be communicated as (\ref{sec:communications})
            within 7 days.
\end{enumerate}

\subsubsection{Transitions}
Members elected in the March election shall assume their office on May 1st.\
Those elected in a by-election assume their office 7 days after the election or
as soon as their predecessor resigns, whichever is earlier. Outgoing Officers
may begin the transition and onboarding of their successors immediately.

\subsection{Constitutional Amendments}
Amendments to the Constitution may be proposed by any Member to the Council. If
at least half ($50\%$) of the Council endorses the amendment, it may be
presented to the general membership. All of~\ref{sec:voting} applies, except
that amendments must be approved by at least a two-thirds majority ($66\%$).

\subsubsection{Cosmetic Changes}
Changes that do not affect the semantics and substance of the Constitution do
not require a vote of the general membership (thereby
circumventing~\ref{sec:voting}). The Council may immediately make these
amendments.

\subsubsection{Version Control}
Changes must be date-stamped and logged through version control
(\href{https://github.com/csgsbs/constitution}{\texttt{github.com/csgsbs/constitution}}).

\subsection{Liability}
The Guild assumes no liability for (in)direct actions of the Department or
individual members of the Guild.